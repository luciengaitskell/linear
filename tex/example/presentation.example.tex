\documentclass{beamer}
\usepackage[utf8]{inputenc}
\usepackage{amsmath}
\usepackage{amssymb,latexsym,amsmath,epsfig,amsthm}


\title{Teaching Tex}
\author{Ms. Walsh}
\date{October 2020}

\begin{document}

\maketitle

\begin{frame}{Introduction}

\pause Hello! This is a teaching document.

\pause Will this create space?

No, \pause but this will\\

%This commands makes extra space depending on what number you use.
\vspace{.2in}

yeah

\end{frame}

\begin{frame}{Math Stuff}

Let's type an equation: $x^2 + y^2 = 12$\\

Let's type two equations:

\begin{align*}
3x + y &= 7\\
9x + 2y &= 8
\end{align*}

Let's type a matrix:

$$\begin{bmatrix}

1 & 0\\
0 & 1

\end{bmatrix}$$

Let's do some matrix addition:

$$\begin{bmatrix}
1 & 0\\
0 & 1
\end{bmatrix} +
\begin{bmatrix}
2 & 3\\
-4 & 5
\end{bmatrix} =
\begin{bmatrix}
3 & 3\\
-4 & 6
\end{bmatrix}$$

\end{frame}

\begin{frame}{More math stuff}

Let's do some matrix multiplication:

$$\begin{bmatrix}
1 & 0\\
0 & 1
\end{bmatrix}
\left(
\begin{bmatrix}
2 & 3\\
-4 & 5
\end{bmatrix} 
\begin{bmatrix}
3 & 3\\
-4 & 6
\end{bmatrix}\right)$$

Let's make an augmented matrix:\\

$$\begin{bmatrix}
4 & 5 & 6 & | & 7\\
3 & 4 & 8 & | & 9
\end{bmatrix}$$

\end{frame}

\begin{frame}{Theorems etc.}

\begin{Theorem}[Ms. Walsh's Theorem]
If $x$ is a potato, then so is $y$.
\end{Theorem}

\begin{Definition}
A \emph{potato} is a friend.
\end{Definition}

\begin{Example}
A sweet potato is an example of a potato.
\end{Example}

\end{frame}

\begin{frame}{Miscellaneous}

$a_ij \neq a_{ij}$\\
$x^12 + y^12 \neq x^{12} + y^{12}$

\begin{itemize}
    \item hello
    \item hello again
\end{itemize}

\begin{enumerate}
    \item hello
    \item hello again
\end{enumerate}

$\frac{1}{2}$\\

$\dfrac{1}{2}$
\end{frame}

\end{document}
